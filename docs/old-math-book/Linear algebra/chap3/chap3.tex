\chapter{More Matrix Operations}
\minitoc
Now we are ready to consider these operations. We provide our own implementation and Jlapack hookup. There is no point to reinvent the wheel if there is a high quality package available. Better algorithms are added over time, such as divide-and-conquer, parallel. 

Template methods for iterative methods.

Estimate run time $O(n^3)$

\section{Matrix Norms}
There are many norms

\section{Matrix Decompositions}
We consider only the commonly used decompositions.

\subsection{LU Decomposition}
Pivoting: no partial or full.

Full LU decomposition is numerical stable for well conditioned matrices.

Lapack SGETRF

\subsection{Cholesky Decomposition}
If a matrix is symmetric, then the LU decomposition can be simplified
\[A = L * D * L^T\]
where L is an unit lower triangular matrix and D is diagonal. If A is positive definite, then we can absorb D into L by taking the square root of the cells of D, \[A= L * L^T\]
Numerically, Cholesky decomposition takes a shortcut using the above expression and thus save time. If A is not positive definite, in theory we still have the first expression, but due to pivoting we can't use it numerically. The best shot is to make D is block diagonal with 2 X 2 blocks.
\subsection{QR Decomposition}
Using householder transformation.

\subsection{Singular Value Decomposition(SVD)}
Do we care symmetric?

Polar decomposition of a matrix A is A = PQ, where P is symmetric positive semidefinite and Q is orthogonal. We can derive this from the SVD.

\subsection{Eigen Decomposition}
This part has several applications:
\begin{itemize}
\item find eigenvalues only
\item find eigenvalues and eigenvectors
\item find eigenvalues, eigenvectors, and the transformation matrix.
\end{itemize}
If the matrix is symmetric, we could use householder matrix to transform it to a tridiagonal matrix, then use QL + implicit shift.

If the matrix is nonsymmetric, use householder to transform it to a Hessenberg form, then transform it to a Schur form.

\section{Determinant}
We could use LU decomposition to compute Determinant for general matrices.

\section{Linear Equation Systems}
Use LU decomposition to solve. pivoting.

general solver()

--> LU

--> cholesky

--> Iterative

Another method is iterative method.
Here are the two packages:

SPARSKIT: http://www-users.cs.umn.edu/~saad/software/

Templates: http://www.netlib.org/templates/
\section{Matrix Inverse and Condition Number}
The inverse of a matrix can be deducted from the linear solver.
The condition number is defined as \[\kappa = {\frac{1}{|A||A^{-1}}|}\]
condition number estimate.

\section{Testing}
test different compositions.
\subsection{Correctness and Error Bound}
\subsection{Algorithm Stability}
\subsection{Performance}
In general, there are two places where we can optimize:

     *      1. time spent to access elements

     *      2. time spent to calculate.

Java array bound checking.

Test the difference between:

Matrix and double[][]

and 

Vector and double[]

for getter/setter.
\section{Reference}

Java implementations:
\begin{itemize}
\item JLapack: http://www.netlib.org/lapack/
\item Colt implementation:
\item Jsci implementation:
\item ojAlgo
\item Jama
\item Jampack
\item UJMP: http://www.ujmp.org/
\end{itemize}

Other implementations:
\begin{itemize}
\item Aglib: for C++, .net
\item newmat: a good C++ implementation
\item BLAS: http://www.netlib.org/blas/
\item BOOST: http://www.boost.org/ 
\item PLapack C implementation: http://www.cs.utexas.edu/~plapack/
\end{itemize}

BLAS

\begin{thebibliography}{99}

% >>>>>>>>> Book examples <<<<<<<<<
\bibitem{CarpenterBOOK} Carpenter, R.H.S., {\textit Movements of the Eyes},
 2nd Edition, Pion Publishing, 1988.

\bibitem{FranklinBOOK} Franklin, G.F., Powel, J.D., Workman, M.L.,
{\textit Digital Control of Dynamic Systems}, Second Edition,
Addison-Wesley, 1990.

% >>>>>>>>> Conference Proceedings Example <<<<<<<<<
\bibitem{OhICRA1998} Oh, P.Y., Allen, P.K., ``Design a Partitioned
 Visual Feedback Controller,'' {\textit IEEE Int Conf Robotics
 \& Automation}, Leuven, Belgium, pp. 1360-1365 5/98

% >>>>>>>>> Journal Example <<<<<<<<<<<<<<<<<<<<<<<<
\bibitem{OhTRA2001} Oh, P.Y., Allen, P.K., ``Visual Servoing
 by Partitioning Degrees of Freedom,'' {\textit IEEE Trans on 
 Robotics \& Automation}, V17, N1, pp. 1-17, 2/01

\end{thebibliography}
