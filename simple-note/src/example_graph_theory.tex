% Preamble
\documentclass[12pt]{simple_doc}

% Packages
\usepackage{simple_note}
\usepackage[UTF8]{ctex} % use chinese
\usepackage{amsmath}
\usepackage{enumitem}

% Document
\begin{document}
    \exerheader{山东大学计算机科学系}{图算法考试}{\today}{ctbBlueDark}

    \begin{cbstripe}{问题一【30分】}{ctbGreenDark}{ctbGreenLight}
        试证明下述命题:
        \begin{enumerate}
            \item 设$n$阶完全图$G=(V, E),\ V_1 \subset V,\ \big | V_1 \big | = n_1$, 且
                $T_1=(V_1, E_1)$是以$V_1$为顶点集的树,
                则在图$G=(V, E)$中以$T_1$为其子树的生成树数目为$n_1\cdot n^{n-n_1 -1}$。
            \item 假设$G=(V, E)$是不可分离的平面图,且每个面都不是三角形,则图$G$的顶点连通性
                $C(G) < 4$。
        \end{enumerate}
    \end{cbstripe}

    \begin{cbstripe}{问题二【30分】}{ctbPurpleDark}{ctbPurpleLight}
        试回答下述问题:
        \begin{enumerate}
            \item 求解图上最短路问题的狄克斯特拉(Dijkstra)算法,对有向图弧的权函数之值有什么要求?
                试举例说明这种要求的必要性?最小生成树算法也有这样要求吗?为什么?
            \item 假设$G=(V, E)$为二分图,其中$V=X \cup Y, X \cap Y = \emptyset$。若$G$中无关
                于顶点集$X$的完全匹配,试找出一顶点子集$A \subset X$,满足$|A| > \big |N(A)\big |$,
                其中$N(A)$为顶点子集的邻域。并举例说明之。
        \end{enumerate}
    \end{cbstripe}

    \begin{cbstripe}{问题三【20分】}{ctbPinkDark}{ctbPinkLight}
        试构造前缀三元$(0, 1, 2)$优化代码,使其代码字具有概率为:
        \begin{equation*}
            \{0.2,\ 0.18,\ 0.12,\ 0.1,\ 0.1,\ 0.08,\ 0.06,\ 0.06,\ 0.06,\ 0.04\}。
        \end{equation*}
    \end{cbstripe}

    \begin{cbstripe}{问题四【20分】}{ctbHoneyDark}{ctbHoneyLight}
        设$G=(V, E)$不是完全图,对任意俩顶点$u, v \in V$,$P(u, v)$表示图$G$中连接顶点$u$和$v$且
        内部顶点互不相交的通路的最大条数。试证明:$P(u, v)$的最小值在$G$中两个不邻接顶点上达到,即
        \begin{equation*}
            \min_{u, v \in G} P(u, v) = \min_{u\ \textit{nadj}\ v} P(u, v)。
        \end{equation*}

    \end{cbstripe}
\end{document}
