% https://bln364.com/finite-state-machine-with-latex/
% https://www3.nd.edu/~kogge/courses/cse30151-fa17/Public/other/tikz_tutorial.pdf

% Preamble
\documentclass[12pt]{simple_doc}

% Packages
\usepackage{simple_note}

\usepackage{fontspec}
\setmonofont{Inconsolata} % change source code font
%\setmainfont{Times New Roman}
\usepackage[default,regular,semibold,t1]{sourceserifpro}

\usepackage{programming}

% Document
\begin{document}
    \exerheader{Programming}{FSM}{\today}{ctbBlueDark}

    \begin{cbstripe}{Problem}{ctbGreenDark}{ctbGreenLight}
        \textit{Consider the following deterministic automation:}

        \begin{center}
        \begin{tikzpicture}
            \node[state, fill=ctbHoneyLight, initial, initial where=left] (n1) {$1$};
            \node[state, right of=n1] (n2) {$2$};
            \node[state, fill=ctbPinkLight, accepting, right of=n2] (n3) {$3$};

            \draw (n1) edge[loop above] node{a} (n1)
                  (n2) edge[loop above] node{a} (n2)
                  (n3) edge[loop above] node{a} (n3)
                  (n1) edge[above] node{b} (n2)
                  (n2) edge[above] node{b} (n3)
                  (n3) edge[bend left, below] node{b} (n1);

        \end{tikzpicture}
        \end{center}

        \textit{Write a program that accepts a string of characters on a single line and,
            processing characters of that string left to right, determines whether that
            string is accepted by that deterministic automation(In which case it prints
            "accept") or rejected(in which case it prints "reject") } \medskip

        Sample input: $abab$\\
        Expected output: $accept$ \\

        Sample input: $ab$\\
        Expected output: $reject$

    \end{cbstripe}

    First, we create the transitions from the graph
    % set line # to be inside frame, not on margins
    \begin{minted}{python}
transitions = {
    1: {'a': 1, 'b': 2},
    2: {'a': 2, 'b': 3},
    3: {'a': 3, 'b': 1},
}
    \end{minted}

    Then we collect the terminal states
    \begin{minted}{python}
terminals = {3}
    \end{minted}

    The automation in fact works with any transitions, it starts with the initial state
    and then traverse the graph.
    \begin{minted}{python}
def process(s: str, init_state, transitions: dict, terminals: set):
    state = init_state
    for c in s:
        tr = transitions[state]
        if c not in tr: return False
        state = tr[c]

    return state in terminals
    \end{minted}

    This return $true$ or $false$, so we need an extra step to get sample outputs
    \begin{minted}{python}
print("accept" if process('abab', 1, transitions, terminals) else "reject")
print("accept" if process('ab', 1, transitions, terminals) else "reject")
    \end{minted}
\end{document}
