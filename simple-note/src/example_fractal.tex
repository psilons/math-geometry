%! Author = Wonderland
%! Date = 10/28/2021

% Preamble
\documentclass[12pt]{simple_doc}

% Packages
\usepackage{simple_note}
\usepackage{tikz}

% see test/geometry/fractal_triangle_test2.tex
% https://github.com/pgf-tikz/pgf/blob/master/doc/generic/pgf/text-en/pgfmanual-en-library-lsystems.tex
% https://latex.org/forum/viewtopic.php?t=17902
\usetikzlibrary{lindenmayersystems}

% Document
\begin{document}
    \exerheader{Geometry}{Fractal}{\today}{ctbBlueDark}

    \begin{cbstripe}{Problem}{ctbPinkDark}{ctbPinkLight}
        We begin with an equilateral triangle with side length 1.
        We divide each side into 3 segments of equal length, and add an equilateral triangle to
        each side using the middle third as a base.
        We then repeat this, to get a third figure.
        If we continue this process forever, what is the area of the resulting figure?

        \tikzset{
            % https://github.com/pgf-tikz/pgf/blob/master/doc/generic/pgf/text-en/pgfmanual-en-library-lsystems.tex
            Koch curve/.style = {
                l-system={
                    rule set={F -> F-F++F-F}, % state transition
                    axiom=F++F++F, % initial state, triangle
                    step=1pt,
                    angle=60,
                    #1
                }
            }
        }

        \centering
        \begin{minipage}{.2\textwidth}
        \centering
        \begin{tikzpicture}
        \draw[Koch curve={order=0,step=70pt}] l-system -- cycle;
        \end{tikzpicture}
        \end{minipage}% no space below, i.e., no new line
        \begin{minipage}{.2\textwidth}
        \centering
        \begin{tikzpicture}
        \draw[Koch curve={order=1,step=70pt/3}] l-system -- cycle;
        \end{tikzpicture}
        \end{minipage}%
        \begin{minipage}{.2\textwidth}
        \centering
        \begin{tikzpicture}
        \draw[Koch curve={order=2,step=70pt/(3^2)}] l-system -- cycle;
        \end{tikzpicture}
        \end{minipage}%
        \begin{minipage}{.2\textwidth}
        \centering
        \begin{tikzpicture}
        \draw[Koch curve={order=3,step=70pt/(3^3)}] l-system -- cycle;
        \end{tikzpicture}
        \end{minipage}%

    \end{cbstripe}

    The shape is called Koch Snowflake, check section 4.9 in the book
    "The Beauty of Fractals Six Different Views", which is in google books.
    Consider the length changes under such transformations.
    Everytime, a side is turned into 4 smaller sides and smaller side X 3 = the larger side.
    So the fractal dimension is \[ \frac{log4}{log3} = 1.26185951\ or\ 4 = 3^{1.26185951}\].

    Initially, or at level 0, it's just the triangle itself.
    The area is $ A = \frac{\sqrt{3}}{4}$ and the number of sides is 3, or $ 3 X 4^0$.

    At leve 1, the area added is
    $ 3\ sides \times \frac{A}{9} = \frac{1}{3}A = \frac{1}{3}(\frac{4}{9})^0 A$.
    The number of sides now is $ 3 \times 4^1 = 12$.

    At leve 2, the area added is
    $ 3 \times 4^1\ sides \times (\frac{1}{9})^2 A = \frac{1}{3}(\frac{4}{9})^1 A$.
    The number of sides now is $ 3 \times 4^2$.

    We compute the total area added from previous side numbers and new triangle area at
    each level.
    To add all levels,

            \[ \sum_{n=1}^{\infty} \frac{1}{3} {\left( \frac{4}{9} \right)}^{n-1} A
            =  \frac{1}{3}  \sum_{n=1}^{\infty} \left(\frac{4}{9}\right)^{n-1} A
            = \frac{1}{3} \times \frac{1}{1 - \frac{4}{9}} A
            = \frac{3}{5} A \]

    So the total area is
    \[A + \frac{3}{5}A = \frac{8}{5}A = \frac{8}{5} \frac{\sqrt{3}}{4} = \frac{2\sqrt{3}}{5}\].
\end{document}
