% Preamble
\documentclass[12pt]{simple_doc}

% Packages
\usepackage{simple_note}
\usepackage{tkz-euclide}

\usepackage{gensymb} % for 90 degree symbol

\usepackage{amsmath} % to remove equation numbers
\usepackage{amssymb}

\usepackage[default,regular,semibold,t1]{sourceserifpro}
\usepackage[T1]{fontenc}

\usepackage{enumitem}

% Document
\begin{document}
    \exerheader{Math}{Geometry}{\today}{ctbRedDark}

    \pgfmathsetlengthmacro{\b}{-17/24} % see end for these numbers
    \pgfmathsetlengthmacro{\h}{sqrt(8*8-\b * \b)}

    \begin{cbstripe}{Problem}{ctbGreenDark}{ctbGreenLight}
        $BE$ and $CF$ are angle bisectors that meet at I as shown below.
        $CE = 4$, $AE = 6$, and $AB = 8$.
        \begin{enumerate}[label={(\alph*)}]
            \item Prove that $\angle EIC = 90\degree - \displaystyle\frac{\angle A}{2}$
            \item Find $BC$
            \item Find $BF$
        \end{enumerate}

        \begin{center}
        \begin{tikzpicture}[scale=1]
            \coordinate (A) at (20*\b,20*\h);
            \coordinate (B) at (0, 0);
            \coordinate (C) at (16/3*20 pt , 0);
            \tkzDrawPolygon[thick](A, B, C);
            \tkzLabelPoints[above](A)
            \tkzLabelPoints[left](B)
            \tkzLabelPoints[right](C)

            \tkzDefLine[bisector](A,B,C) \tkzGetPoint{X};
            \tkzInterLL(B,X)(A,C) \tkzGetPoint{E};
            \tkzLabelPoints[above right](E)
            \draw (B) -- (E);

            \tkzDefLine[bisector](A,C,B) \tkzGetPoint{Y};
            \tkzInterLL(C,Y)(A,B) \tkzGetPoint{F};
            \tkzLabelPoints[left](F)
            \draw (C) -- (F);

            \tkzInterLL(C,F)(B,E) \tkzGetPoint{I};
            \tkzLabelPoints[above](I)

        \end{tikzpicture}
        \end{center}
    \end{cbstripe}

    (a) Because $\angle EIC$ is an external angle of the triangle $\triangle EIC$,
    \begin{equation*}
		\begin{aligned}
            \angle EIC &= \angle IBC + \angle ICB\\
                       &= \frac{\angle B}{2} + \frac{\angle C}{2}\\
        \end{aligned}
	\end{equation*}
    Hence,
    \begin{equation*}
		\begin{aligned}
            \frac{\angle A}{2} + \angle EIC
                &= \frac{\angle A}{2} + \frac{\angle B}{2} + \frac{\angle C}{2}\\
                &= \frac{\angle A + \angle B + \angle C}{2}\\
                &= \frac{180\degree}{2}\\
                &= 90\degree
        \end{aligned}
	\end{equation*}
    \medskip

    (b) Draw a line passing E and parallel to BC, intersect with $AB$ at $M$.
    \begin{center}
        \begin{tikzpicture}[scale=1]
            \coordinate (A) at (20*\b,20*\h);
            \coordinate (B) at (0, 0);
            \coordinate (C) at (16/3*20 pt , 0);
            \tkzDrawPolygon[thick](A, B, C);
            \tkzLabelPoints[above](A)
            \tkzLabelPoints[left](B)
            \tkzLabelPoints[right](C)

            \tkzDefLine[bisector](A,B,C) \tkzGetPoint{X};
            \tkzInterLL(B,X)(A,C) \tkzGetPoint{E};
            \tkzLabelPoints[above right](E)
            \draw (B) -- (E);

%            \tkzDefLine[bisector](A,C,B) \tkzGetPoint{Y};
%            \tkzInterLL(C,Y)(A,B) \tkzGetPoint{F};
%            \tkzLabelPoints[left](F)
%            \draw (C) -- (F);

            \tkzDefLine[parallel=through E](B,C) \tkzGetPoint{Z};
            \tkzInterLL(E,Z)(A,B) \tkzGetPoint{M};
            \tkzLabelPoints[left](M)
            \draw[dashed,red] (M) -- (E);
            \tkzMarkAngle[size=5mm,color=red,mark=|](E,B,M);
            \tkzMarkAngle[size=5mm,color=red,mark=|](C,B,E);
            \tkzMarkAngle[size=5mm,color=red,mark=|](M,E,B);

            \tkzLabelSegment[above](M,E){$x$}
            \tkzLabelSegment[right](M,B){$x$}
            \tkzLabelSegment[below](B,C){$y$}
            \tkzLabelSegment[above right](A,E){$6$}
            \tkzLabelSegment[above right](E,C){$4$}
        \end{tikzpicture}
    \end{center}
    Because $\angle MBE = \angle EBC = \angle MEB$, $ME = MB$. Because
    $\triangle AME \sim \triangle ABC$,
    \begin{equation*}
        \begin{aligned}
            \frac{AM}{AB} &= \frac{AE}{AC} \\
            \frac{8 - x}{8} &= \frac{6}{10} \\
            10x &= 32\\
            x &= 3.2
        \end{aligned}
	\end{equation*}
    Then
    \begin{equation*}
        \begin{aligned}
            \frac{ME}{BC} &= \frac{AE}{AC} \\
            \frac{x}{y} &= \frac{6}{10} \\
            6y &= 32\\
            y &= \frac{16}{3}
        \end{aligned}
	\end{equation*}
    \medskip

    (c) Draw a line through F and parallel to BC, intersect with AC at N.
    \begin{center}
        \begin{tikzpicture}[scale=1]
            \coordinate (A) at (20*\b,20*\h);
            \coordinate (B) at (0, 0);
            \coordinate (C) at (16/3*20 pt , 0);
            \tkzDrawPolygon[thick](A, B, C);
            \tkzLabelPoints[above](A)
            \tkzLabelPoints[left](B)
            \tkzLabelPoints[right](C)

%            \tkzDefLine[bisector](A,B,C) \tkzGetPoint{X};
%            \tkzInterLL(B,X)(A,C) \tkzGetPoint{E};
%            \tkzLabelPoints[above right](E)
%            \draw (B) -- (E);

            \tkzDefLine[bisector](A,C,B) \tkzGetPoint{Y};
            \tkzInterLL(C,Y)(A,B) \tkzGetPoint{F};
            \tkzLabelPoints[left](F)
            \draw (C) -- (F);

            \tkzDefLine[parallel=through F](B,C) \tkzGetPoint{Z};
            \tkzInterLL(F,Z)(A,C) \tkzGetPoint{N};
            \tkzLabelPoints[above right](N)
            \draw[dashed,red] (N) -- (F);
            \tkzMarkAngle[size=5mm,color=red,mark=|](N,C,F);
            \tkzMarkAngle[size=5mm,color=red,mark=|](F,C,B);
            \tkzMarkAngle[size=5mm,color=red,mark=|](C,F,N);

            \tkzLabelSegment[above](N,F){$u$}
            \tkzLabelSegment[right](N,C){$u$}
            \tkzLabelSegment[left](F,B){$v$}
            \tkzLabelSegment[left](A,B){$8$}
            \tkzLabelSegment[below](B,C){$\displaystyle\frac{16}{3}$}
        \end{tikzpicture}
    \end{center}
    Because $\angle NCF = \angle FCB = \angle NFC$, $FN = NC$. Because
    $\triangle AFN \sim \triangle ABC$,
    \begin{equation*}
        \begin{aligned}
            \frac{AN}{AC} &= \frac{FN}{BC} \\
            \frac{10 - u}{10} &= \frac{u}{\frac{16}{3}} = \frac{3u}{16} \\
            46u &= 160\\
            u &= \frac{80}{23}
        \end{aligned}
	\end{equation*}
    Then
    \begin{equation*}
        \begin{aligned}
            \frac{AF}{AB} &= \frac{FN}{BC} \\
            \frac{8-v}{8} &= \frac{u}{\frac{16}{3}} = \frac{3u}{16} = \frac{15}{23} \\
            v &= \frac{64}{23}
        \end{aligned}
	\end{equation*}
    \medskip

    (d) To draw the triangle, we need to figure out the coordinate of A
    \begin{center}
        \begin{tikzpicture}[scale=1]
            \coordinate (A) at (20*\b,20*\h);
            \coordinate (B) at (0, 0);
            \coordinate (C) at (16/3*20 pt , 0);
            \tkzDrawPolygon[thick](A, B, C);
            \tkzLabelPoints[above](A)
            \tkzLabelPoints[above right](B)
            \tkzLabelPoints[right](C)

            \tkzDrawLine[altitude,dashed,red](B,A,C) \tkzGetPoint{P}; % from A to BC
            \tkzLabelPoints[left](P);
            \draw[dashed,red] (P) -- (B);

            \tkzLabelSegment[left](A,P){$h$}
            \tkzLabelSegment[below](P,B){$w$}
            \tkzLabelSegment[above right](A,B){$8$}
            \tkzLabelSegment[above right](A,C){$10$}
            \tkzLabelSegment[below](B,C){$\displaystyle\frac{16}{3}$}
        \end{tikzpicture}
    \end{center}
    With the 2 right triangles $\triangle APB$ and $\triangle APC$, the followings are true
    \begin{equation*}
        \begin{aligned}
            h^2 + w^2 &= 8^2 = 64 \\
            h^2 + (\frac{16}{3} + w)^2 &= 10^2 = 100\\
        \end{aligned}
	\end{equation*}
    Subtracting these 2 equations,
    \begin{equation*}
        \begin{aligned}
            (\frac{16}{3} + w)^2 - w^2 &= 10^2 = 36\\
            w &= - \frac{17}{24} = 0.62963\\
            h &= \sqrt{64 - w^2} = 7.96858
        \end{aligned}
	\end{equation*}
\end{document}
