% Preamble
\documentclass[12pt]{simple_doc}

% Packages
\usepackage{simple_note}

\usepackage{fontspec}
\setmonofont{Inconsolata} % change source code font
%\setmainfont{Times New Roman}
\usepackage[default,regular,semibold,t1]{sourceserifpro}

\usepackage{programming}
\usepackage{enumitem}

% Document
\begin{document}
    \exerheader{Programming}{FSM}{\today}{ctbBlueDark}
    \begin{cbstripe}{Problem: LC67}{ctbPinkDark}{ctbPinkLight}
        A valid \textbf{number} can be split up into(in order):
        \begin{enumerate}
            \item a \textbf{decimal number} or an \textbf{integer}.
            \item (optional) An \monoquote{e} or \monoquote{E}, followed by an integer.
        \end{enumerate}
        A \textbf{decimal number} can split up into(in order):
        \begin{enumerate}
            \item (optional) A sign character, either \monoquote{+} or \monoquote{-}.
            \item One of the following formats:
                \begin{enumerate}
                    \item One or more digits, followed by \monoquote{.}.
                    \item One or more digits, followed by \monoquote{.}, followed by one more digits
                    \item A dot \monoquote{.}, followed by one or more digits.
                \end{enumerate}
        \end{enumerate}
        An \textbf{integer} can be split up into:
        \begin{enumerate}
            \item (optional) A sign character, either \monoquote{+} or \monoquote{-}.
            \item one or more digits.
        \end{enumerate}
        For example, all the following are valid numbers:\medskip \\
        \texttt{
            \hspace*{1em} "2", "0089", "-0.1", "+3.14", "4.", "-.9", "2e10", "-90E3", "3e+7",\\
            \hspace*{1em} "+6e-1", "53.5e93", "-123.456e789"} \medskip \\
        while the following are not valid numbers: \medskip\\
        \hspace*{1em} \texttt{"abc", "1a", "1e", "e3", "99e2.5", "--6", "-+3", "95a54e53"} \medskip \\
        Given a string \texttt{s} return \texttt{true} if \texttt{s} is a valid \textbf{number} or
        \texttt{false} otherwise.

        \medskip
        \textbf{Example 1:}
        \begin{enumerate}[topsep=0pt,itemsep=-1ex,partopsep=1ex,parsep=1ex]
            \item[] Input: s = "0"
            \item[] Output: true
        \end{enumerate}
        \textbf{Example 2:}
        \begin{enumerate}[topsep=0pt,itemsep=-1ex,partopsep=1ex,parsep=1ex]
            \item[] Input: s = "e"
            \item[] Output: false
        \end{enumerate}
        \textbf{Example 3:}
        \begin{enumerate}[topsep=0pt,itemsep=-1ex,partopsep=1ex,parsep=1ex]
            \item[] Input: s = "."
            \item[] Output: false
        \end{enumerate}
        \textbf{Example 4:}
        \begin{enumerate}[topsep=0pt,itemsep=-1ex,partopsep=1ex,parsep=1ex]
            \item[] Input: s = ".1"
            \item[] Output: true
        \end{enumerate}
    \end{cbstripe}

    Here is the finite state automata based on the problem description. The yellow node is the
    start, and the pink nodes are terminals.
    \begin{center}
    \begin{tikzpicture}
        \node[state, fill=ctbHoneyLight, initial, initial where=left] (start) {$start$};
        \node[state, below right of=start] (sign1) {$sign1$};
        \node[state, below of=sign1] (dot) {$dot$};
        \node[state, fill=ctbPinkLight, accepting, above right of=sign1] (D1) {$D1$};
        \node[state, fill=ctbPinkLight, accepting, right of=dot] (D2) {$D2$};
        \node[state, right of=D2] (Exp) {$Exp$};
        \node[state, below of=Exp] (sign2) {$sign2$};
        \node[state, fill=ctbPinkLight, accepting, right of=sign2] (D3) {$D3$};

        \draw (start) edge[above right] node{+/-} (sign1)
              (start) edge[left] node{.} (dot)
              (start) edge[above] node{d} (D1)
              (sign1) edge[above left] node{d} (D1)
              (sign1) edge[right] node{.} (dot)
              (dot) edge[above] node{d} (D2)
              (D1) edge[loop above] node{d} (D1)
              (D1) edge[right] node{.} (D2)
              (D1) edge[above right] node{E/e} (Exp)
              (D2) edge[above] node{E/e} (Exp)
              (D2) edge[loop below] node{d} (D2)
              (Exp) edge[right] node{+/-} (sign2)
              (Exp) edge[above right] node{d} (D3)
              (sign2) edge[above] node{d} (D3)
              (D3) edge[loop right] node{d} (D3);

    \end{tikzpicture}
    \end{center}

    From the diagram, we can create the corresponding transitions.
    \begin{minted}{python}
transitions = {
    'start': {'.': 'dot', '+': 'sign1', '-': 'sign1', 'd': 'D1'},
    'sign1': {'.': 'dot', 'd': 'D1'}, # base sign
    'dot': {'d': 'D2'}, # base dot
    'D1': { 'd': 'D1', '.': 'D2', 'E': 'Exp', 'e': 'Exp'}, # integer
    'D2': {'d': 'D2', 'E': 'Exp', 'e': 'Exp'}, # float
    'Exp': {'+': 'sign2', '-': 'sign2', 'd': 'D3'},
    'sign2': {'d': 'D3'}, # exponent sign
    'D3': {'d': 'D3'} # exponent
}
    \end{minted}

    The terminals are
    \begin{minted}{python}
terminals = {'D1', 'D2', 'D3'}
    \end{minted}

    The automata program is
    \begin{minted}{python}
state = 'start'
for c in s:
    if c.isdigit(): c = 'd'
    tr = transitions[state]
    if c not in tr: return False
    state = tr[c]

return state in terminals
    \end{minted}

\end{document}
